%% 2.5 %%
\section{Subsequences and the Bolzano-Weierstrass Theorem}
\setcounter{exercise}{0}

\bx{
Suppose we have a convergent sequence with limit $l$.
Then given any $\epsilon > 0$, we can always find $N: n \geq N$ such that
$\abs{a_n - l} < \epsilon$.
For any subsequence of $(a_n)$, $(a'_m)$, any element of this subsequence,
call it $a'_k$ will be from some $a_n$ in the original sequence,
where $n \geq k$. So we can choose $N$ from earlier, and for $m \geq N$
we will have $\abs{a'_m - l} < \epsilon$.
}

\bx{
\ea{
	\item Define
\begin{align}
	s_i &= \sum_{j=1}^{i} a_j  \\
	b_i &= \sum_{k=1}^{i} a_{n_k},
\end{align}
where the series regrouping $a_i$ is divided into groups of $n_1, n_2, \dots$. Then $b_i$ is a subsequence of $s_n$, which means they converge to the same limit, namely $L$ in this case.
	\item Our proof does not apply to that example because that series did not converge in the first place.
}
}

\bx{
\ea{
	\item Consider
	\begin{equation}
		a_n = \begin{cases}
			\sum_{i=1}^n \frac{1}{2^i}, &n \text{ odd } \\
			\frac{1}{2^i}	, &n \text{ even}
		\end{cases}
	\end{equation}
	Then we have that $b_n = a_{2n-1}$ converges to 1 and $c_n = a_{2n}$ converges to 0.
	\item A monotone sequence that diverges means that sequence is not bounded. Thus, every subsequence will also be unbounded and thus impossible to be convergent.
	\item Consider the sequence
	\begin{equation}
		\{1, 1, \frac{1}{2}, 1, \frac{1}{2}, \frac{1}{3}, 1, \frac{1}{2}, \frac{1}{3}, \frac{1}{4}, \dots\}
	\end{equation}
	\item Consider
	\begin{equation}
		a_n = \begin{cases}
			2^i, &n \text{ odd } \\
			\frac{1}{2^i}	, &n \text{ even}
		\end{cases}
		\label{eq:convdiv}
	\end{equation}
	\item By Bolzano-Weierstrass, since we have a subsequence that is bounded,
	we know we can find a convergent subsequence within this subsequence that converges.
}
}

\bx{
	\AFSOC $(a_n)$ converges to $b \neq a$. Then we have that $\abs{a_n - b}$ can be arbitrarily small. But this implies that every subsequence will also converge to $b$, which is a contradiction.

	\AFSOC $(a_n)$ does not converge. Then since $(a_n)$ is bounded, we must have an infinite number of elements in
	two different $\epsilon$-neighborhoods. But this would imply we have convergent subsequences to different limits,
	which contradicts the original problem statement.

	Therefore, we conclude $(a_n)$ converges to $a$.
}

\bx{
Consider $\abs{b^n}$. Since $\abs{b}<1$, we have that $\abs{b^n}$ is a decreasing sequence that is bounded below by 0, so we have
\begin{equation*}
	\abs{b} > l \geq 0.
\end{equation*}
We notice that $\abs{b^{2n}}$ is a subsequence that also converges to $L$, and since $\abs{b^{2n}} = \abs{b}^2$, by the Algebraic Limit Theorem, we have that $\abs{b^{2n}} \rightarrow l^2 = l \implies l = 0$. Since $\abs{b^n} \rightarrow 0$, we conclude $b^n \rightarrow 0$.
}

\bx{
We have $s = \sup S$, which means for any $\epsilon > 0$,
\begin{align*}
	\exists x : s - \epsilon < x \in S < a'_n \\
	\epsilon > \abs{s - a'_n} = \abs{a'_n - s}
\end{align*}
where $a'_n$ is an element of the infinite subsequence of $a_n:a_n > x \in S$.
}